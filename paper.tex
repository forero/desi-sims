\documentclass[12pt]{article}
\title {DESI simulation software}
\begin{document}
\maketitle
\begin{abstract}
The simulation software for the Dark Energy Spectroscopic Instrument
(DESI) can be used for large-scale modelling of the experiment over
its 5-year lifetime under different conditions.
This framework integrates a large number of packages including the
noiseless generation of object spectra, the simulation of CCD
response and the final extraction of results in the spectroscopic
pipeline. 
In this paper we discuss the DESI simulation infrastructure, including
the standard survey strategy, instrument parameters, object
simulation, the CCD response to finalize with the expected
DESI performance to create a redshift catalog including at least 35
Million objects. 
We also describe the framework for software validation,
performance testing and output vaildation against known results. 
\end{abstract}

\tableofcontents
\section{Introduction}

\section{DESI Offline software overview}

\section{DESI simulation overview}

\section{Mock Data Generation Overview}

\subsection{DESI program}

\subsection{BGS program}

\subsection{MWS program}

\section{DESI Detector Overview}

\subsection{Telescope}
\subsection{Fiber positioners}
\subsection{Spectrographs}
\subsection{CCDs}

\section{DESI Spectroscopic Extraction Overview}

\section{End-to-End Simulation}

\subsection{Simulation Input}

\subsection{Truth information}

\subsection{Simulation Initialization}

\subsection{Night-by-night scheduling}

\subsection{Data Products}

\subsection{Visualization}

\section{Fast Simulations}

\section{Validation}
\subsection{Automated Testing}
\subsection{Computing Perfomance Benchmarking}
\subsection{Astrophysical Validation}


\section{Summary and Conclusions}

\end{document}
